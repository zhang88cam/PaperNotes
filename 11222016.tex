%%%%%%%%%%%%%%%%%%%%%%%%%%%%%%%%%%%%%%%%%
% Article Notes
% LaTeX Template
% Version 1.0 (1/10/15)
%
% This template has been downloaded from:
% http://www.LaTeXTemplates.com
%
% Authors:
% Vel (vel@latextemplates.com)
% Christopher Eliot (christopher.eliot@hofstra.edu)
% Anthony Dardis (anthony.dardis@hofstra.edu)
%
% License:
% CC BY-NC-SA 3.0 (http://creativecommons.org/licenses/by-nc-sa/3.0/)
%
%%%%%%%%%%%%%%%%%%%%%%%%%%%%%%%%%%%%%%%%%

%----------------------------------------------------------------------------------------
%	PACKAGES AND OTHER DOCUMENT CONFIGURATIONS
%----------------------------------------------------------------------------------------

\documentclass[
10pt, % Default font size is 10pt, can alternatively be 11pt or 12pt
a4paper, % Alternatively letterpaper for US letter
onecolumn, % Alternatively onecolumn,twocolumn
portrait % Alternatively portrait,landscape
]{article}

\usepackage{CJKutf8}

%%%%%%%%%%%%%%%%%%%%%%%%%%%%%%%%%%%%%%%%%
% Article Notes
% Structure Specification File
% Version 1.0 (1/10/15)
%
% This file has been downloaded from:
% http://www.LaTeXTemplates.com
%
% Authors:
% Vel (vel@latextemplates.com)
% Christopher Eliot (christopher.eliot@hofstra.edu)
% Anthony Dardis (anthony.dardis@hofstra.edu)
%
% License:
% CC BY-NC-SA 3.0 (http://creativecommons.org/licenses/by-nc-sa/3.0/)
%
%%%%%%%%%%%%%%%%%%%%%%%%%%%%%%%%%%%%%%%%%

%----------------------------------------------------------------------------------------
%	REQUIRED PACKAGES
%----------------------------------------------------------------------------------------

\usepackage[includeheadfoot,columnsep=2cm, left=1in, right=1in, top=.5in, bottom=.5in]{geometry} % Margins

\usepackage[T1]{fontenc} % For international characters
\usepackage{XCharter} % XCharter as the main font

\usepackage{natbib} % Use natbib to manage the reference
\bibliographystyle{apalike} % Citation style

\usepackage[english]{babel} % Use english by default

%----------------------------------------------------------------------------------------
%	CUSTOM COMMANDS
%----------------------------------------------------------------------------------------

\newcommand{\articletitle}[1]{\renewcommand{\articletitle}{#1}} % Define a command for storing the article title
\newcommand{\articlecitation}[1]{\renewcommand{\articlecitation}{#1}} % Define a command for storing the article citation
\newcommand{\doctitle}{\articlecitation\ --- ``\articletitle''} % Define a command to store the article information as it will appear in the title and header

\newcommand{\datenotesstarted}[1]{\renewcommand{\datenotesstarted}{#1}} % Define a command to store the date when notes were first made
\newcommand{\docdate}[1]{\renewcommand{\docdate}{#1}} % Define a command to store the date line in the title

\newcommand{\docauthor}[1]{\renewcommand{\docauthor}{#1}} % Define a command for storing the article notes author

% Define a command for the structure of the document title
\newcommand{\printtitle}{
\begin{center}
\textbf{\Large{\doctitle}}

\docdate

\docauthor
\end{center}
}

%----------------------------------------------------------------------------------------
%	STRUCTURE MODIFICATIONS
%----------------------------------------------------------------------------------------

\setlength{\parskip}{3pt} % Slightly increase spacing between paragraphs

% Uncomment to center section titles
%\usepackage{sectsty}
%\sectionfont{\centering}

% Uncomment for Roman numerals for section numbers
%\renewcommand\thesection{\Roman{section}}
 % Input the file specifying the document layout and structure

%----------------------------------------------------------------------------------------
%	ARTICLE INFORMATION
%----------------------------------------------------------------------------------------

\articletitle{ TLBO strategy} % The title of the article
\articlecitation{\cite{Zhang_Parameter_2015}} % The BibTeX citation key from your bibliography

\datenotesstarted{November 22, 2016} % The date when these notes were first made
\docdate{\datenotesstarted; rev. \today} % The date when the notes were lasted updated (automatically the current date)

\docauthor{QZ} % Your name

%----------------------------------------------------------------------------------------

\begin{document}
\begin{CJK*}{UTF8}{gbsn}
\pagestyle{myheadings} % Use custom headers
\markright{\doctitle} % Place the article information into the header

%----------------------------------------------------------------------------------------
%	PRINT ARTICLE INFORMATION
%----------------------------------------------------------------------------------------

\thispagestyle{plain} % Plain formatting on the first page

\printtitle % Print the title

%----------------------------------------------------------------------------------------
%	ARTICLE NOTES
%----------------------------------------------------------------------------------------

\section*{Introduction} % Unnumbered section

Compared with GA, PSO and DEA, TLBO has some attractive
 characteristics. First, it employs simple differential operation
 between teacher and students to create new candidate solu-
 tions, as well as to guide the search toward the most promising
 region. Second, TLBO works with real numbers in natural
 manner and avoid complicated generic searching operators
 in GA, and twofold updating strategy in PSO. Third, the
 conventional TLBO only contains one adjustable controlling
 parameter (i.e., population size), which facilitates easy tun-
 ing and implementation, while in GA, PSO and DEA more
 parameters need to be set in an appropriate manner so as to
 guarantee the searching performance.


er so as to
 guarantee the searching performance. Nowadays, TLBO has
 attracted attention and applications in a few elds since its
 birth in 2011 (Patel and Savsani; Crepinsek et al. 2012; Wagh-
 mare 2013). For instance, Rao and Patel (2013c) enhanced
 the conventional TLBO to form M-TLBO (named by the
 authors of this paper) by incorporating three operations, i.e.,
 group learning (number of teachers), adaptive teaching fac-
 tor and self-motivated learning. I-TLBO (Patel and Savsani
 2014a, b) was developed by introducing an additional oper-
 ation, labeled as learning through tutorial operator into the
 above M-TLBO. Application areas cover dynamic economic
 emission dispatch (Niknam et al. 2012), structural optimiza-
 tion (Dede 2013), power system (García Ansola et al. 2012),
 heat exchangers (Rao and Patel 2013b; Patel and Savsani
 2014b), thermoelectric cooler (V enkata Rao and Patel 2013)
 and engineering optimization problems (Y u et al. 2014), etc.,
 which demonstrate the effectiveness and efciency of the
 TLBO-based algorithms.

使用TLBO解决实际问题的一篇文章



%----------------------------------------------------------------------------------------
%	BIBLIOGRAPHY
%----------------------------------------------------------------------------------------

\renewcommand{\refname}{Reference} % Change the default bibliography title

\bibliography{11162016} % Input your bibliography file

%----------------------------------------------------------------------------------------
\end{CJK*}
\end{document}