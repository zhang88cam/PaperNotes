%%%%%%%%%%%%%%%%%%%%%%%%%%%%%%%%%%%%%%%%%
% Article Notes
% LaTeX Template
% Version 1.0 (1/10/15)
%
% This template has been downloaded from:
% http://www.LaTeXTemplates.com
%
% Authors:
% Vel (vel@latextemplates.com)
% Christopher Eliot (christopher.eliot@hofstra.edu)
% Anthony Dardis (anthony.dardis@hofstra.edu)
%
% License:
% CC BY-NC-SA 3.0 (http://creativecommons.org/licenses/by-nc-sa/3.0/)
%
%%%%%%%%%%%%%%%%%%%%%%%%%%%%%%%%%%%%%%%%%

%----------------------------------------------------------------------------------------
%	PACKAGES AND OTHER DOCUMENT CONFIGURATIONS
%----------------------------------------------------------------------------------------

\documentclass[
10pt, % Default font size is 10pt, can alternatively be 11pt or 12pt
a4paper, % Alternatively letterpaper for US letter
onecolumn, % Alternatively onecolumn,twocolumn
portrait % Alternatively portrait,landscape
]{article}

\usepackage{CJKutf8}

%%%%%%%%%%%%%%%%%%%%%%%%%%%%%%%%%%%%%%%%%
% Article Notes
% Structure Specification File
% Version 1.0 (1/10/15)
%
% This file has been downloaded from:
% http://www.LaTeXTemplates.com
%
% Authors:
% Vel (vel@latextemplates.com)
% Christopher Eliot (christopher.eliot@hofstra.edu)
% Anthony Dardis (anthony.dardis@hofstra.edu)
%
% License:
% CC BY-NC-SA 3.0 (http://creativecommons.org/licenses/by-nc-sa/3.0/)
%
%%%%%%%%%%%%%%%%%%%%%%%%%%%%%%%%%%%%%%%%%

%----------------------------------------------------------------------------------------
%	REQUIRED PACKAGES
%----------------------------------------------------------------------------------------

\usepackage[includeheadfoot,columnsep=2cm, left=1in, right=1in, top=.5in, bottom=.5in]{geometry} % Margins

\usepackage[T1]{fontenc} % For international characters
\usepackage{XCharter} % XCharter as the main font

\usepackage{natbib} % Use natbib to manage the reference
\bibliographystyle{apalike} % Citation style

\usepackage[english]{babel} % Use english by default

%----------------------------------------------------------------------------------------
%	CUSTOM COMMANDS
%----------------------------------------------------------------------------------------

\newcommand{\articletitle}[1]{\renewcommand{\articletitle}{#1}} % Define a command for storing the article title
\newcommand{\articlecitation}[1]{\renewcommand{\articlecitation}{#1}} % Define a command for storing the article citation
\newcommand{\doctitle}{\articlecitation\ --- ``\articletitle''} % Define a command to store the article information as it will appear in the title and header

\newcommand{\datenotesstarted}[1]{\renewcommand{\datenotesstarted}{#1}} % Define a command to store the date when notes were first made
\newcommand{\docdate}[1]{\renewcommand{\docdate}{#1}} % Define a command to store the date line in the title

\newcommand{\docauthor}[1]{\renewcommand{\docauthor}{#1}} % Define a command for storing the article notes author

% Define a command for the structure of the document title
\newcommand{\printtitle}{
\begin{center}
\textbf{\Large{\doctitle}}

\docdate

\docauthor
\end{center}
}

%----------------------------------------------------------------------------------------
%	STRUCTURE MODIFICATIONS
%----------------------------------------------------------------------------------------

\setlength{\parskip}{3pt} % Slightly increase spacing between paragraphs

% Uncomment to center section titles
%\usepackage{sectsty}
%\sectionfont{\centering}

% Uncomment for Roman numerals for section numbers
%\renewcommand\thesection{\Roman{section}}
 % Input the file specifying the document layout and structure

%----------------------------------------------------------------------------------------
%	ARTICLE INFORMATION
%----------------------------------------------------------------------------------------

\articletitle{A Continuous-Time Markov Decision Process-Based Method With Application in a Pursuit-Evasion Example} % The title of the article
\articlecitation{\cite{Jia_A_2016}} % The BibTeX citation key from your bibliography

\datenotesstarted{November 17, 2016} % The date when these notes were first made
\docdate{\datenotesstarted; rev. \today} % The date when the notes were lasted updated (automatically the current date)

\docauthor{QZ} % Your name

%----------------------------------------------------------------------------------------

\begin{document}
\begin{CJK*}{UTF8}{gbsn}
\pagestyle{myheadings} % Use custom headers
\markright{\doctitle} % Place the article information into the header

%----------------------------------------------------------------------------------------
%	PRINT ARTICLE INFORMATION
%----------------------------------------------------------------------------------------

\thispagestyle{plain} % Plain formatting on the first page

\printtitle % Print the title

%----------------------------------------------------------------------------------------
%	ARTICLE NOTES
%----------------------------------------------------------------------------------------

\section*{Introduction} % Unnumbered section

连续的Markov决定链解决追逃问题
\\ Pursuit-Evasion 是计算机领域很常见的问题?wiki解释大概猜测,这是图论的问题

\begin{enumerate}
\item  geometric formulation - continuous pursuit-evasion
\item graph formulation - discrete pursuit-evasion (also called graph searching). 
\end{enumerate}

Pursuit-Evasion分为连续和离散两种,分别对应不同的求解方法

\begin{enumerate}
\item Optimal  control  problem 最优控制问题 - 连续
\item Markov链 - 离散
\end{enumerate}

维度的增加会导致搜索空间的指数增加 (最优化问题在解决的方向)
CTMDP可以解决MDP遇到的复杂动态系统无法解决的问题

查论文
 \\ S. Jia, X. Wang, X. Ji, and H. Zhu, “A continuous-time Markov decision process based method on pursuit-evasion problem,” in Proc. 19th
 IFAC   World   Congr.,   vol.   19.   Cape   Town,   South   Africa,   Aug.   2014, pp. 620–625.


A  Motivation

%------------------------------------------------

%\section{Methodology Overview} % Numbered section



%------------------------------------------------

%\section{Results Overview}



%------------------------------------------------

%\section{Discussion/Conclusions Overview}



%------------------------------------------------

\section*{Article Evaluation}
需要仔细思考的问题
\begin{description}
\item[1]  Markov链一般用来解决什么问题 - 随机模型
\item[2]  为什么Pursuit-Evasion是一个普遍问题,作为模型可以用来解决什么实际问题
\item[3]  动态规划和Markov链之间的关系 - 
动态规划解决Markov链的问题

\end{description}



%----------------------------------------------------------------------------------------
%	BIBLIOGRAPHY
%----------------------------------------------------------------------------------------

\renewcommand{\refname}{Reference} % Change the default bibliography title

\bibliography{11162016} % Input your bibliography file

%----------------------------------------------------------------------------------------
\end{CJK*}
\end{document}