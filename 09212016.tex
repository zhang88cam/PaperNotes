\documentclass[a4paper,12pt]{article}
\begin{document}
9/21 - 9/28
\\Notes are written down for all papers that I have read in this week. 
\begin{enumerate}
	\item SSO social spider optimization\cite{Cuevas_A_2013}
	\\Two search agent: male and female, three operators, share the communication
	
	\item SSO-C SSO for constrained optimization\cite{Cuevas_A_2014}
	\\Modify: penalty function, criterion to bias. Provide an application on mechanical problem
	
	\item SMS algorithm\cite{Cuevas_An_2014} has three states: gas, liquid and solid, which are sperated with the proportion iteration numbers. Gas state focus on exploration, liquid is a transit state, while solid focus on exploitation.
	
	\item This paper \cite{Anthony_Ap_2013} indicates the methology of dealing with raw data and try to build a model from it. I cannot fully understand all theories, and needs more investigation on it.
	\\He collected cellular data, and analysis data try to address some patterns in it. There are some parameters need to be considered like assortativity, clustering coefficient,  degree distribution,  mean degree. The popular model like ER random graph cannot resolve assortativity.
	\\Introduced a  physical model: maximum emtropy principle. He also addressed that ER random graph is the result of maximum entropy principle.
	%A new partition function 
	%An agent is for the modeled data. like a virtual vertex to represent the social ties. It knows its own social topology but know nothing about others
	%Establish a difinition of link reciprocity, the protential difference. 
	\\An agent based model and some parameters activity, energy and inverse temperature  \(\beta\) are broght in for the new model.
	\\Hot regime (low \(\beta\)) represents gas state.
	\\Cold regime (high \(\beta\)) represents liquid state.
	
	\item It proposed a self-detect method to rapidly recover from damage for the deployed robot itself. \cite{Cully_Robots_2015}
	\\There is a map of behavior-performace space.
	\\All of these ideas are technically captured via a Gaussian process model, which approximates the performance function with already acquired data, and a Bayesian optimization procedure
	\\ adapting to new environments
	\\The robot selects which behaviours to test by maximizing an information acquisition function that balances exploration (selecting points whose performance is uncertain) and exploitation (selecting points whose performance is expected to be high) (see Supplementary Methods). 
	
	\item It is announced it bridges the gap between practice and theory. Constraint programming relating. \cite{Cooper_Broken_2016} I cannot understand the proving and hard to understand where it can be used for.
	\\The BTP-merging the paper provided could reduce the total number of variable-value assignments.
	
	\item  A new approach to handle queries over incomplete databases. \cite{Libkin_Certain_2016} Skim
	
	\item GSO Group Search Optimizer\cite{He_Group_2009}
	\\The original PS model consists of only producers and scroungers. For the optimization purpose, e.g., avoiding entrapments by local minima, we introduced the rangers
	\\Apply GSO to ANN weight training
	
	\item MatLab toolbox’s default Levenberg–Marquardt algorithm to train.\cite{Vernon_Modeling_2014}
	%Linear activation function
	\\Validate the 16 social traits which relates to 3 factors, cross validate attributes to these 3 factors.
	
	\item CRO chemical reation optimization algorithm implementation. Go through the concepts, but hasn't tried to implemented yet. How to use in the applications. \cite{Lam_Chemical_2012}
	
	\item CRO for grid scheduling problem \cite{Xu_Chemical_2011}
	\\Modify:   only one molecule is generated in the initialization stage; \(central energy buffer\neq 0\); iteration stage was divided into two parts. It is similar to SMS.
	
	\item RCCRO real coded chemical realtion optimaztion \cite{Lam_Real_2012}
	\\Modification
	\\Solution representation, Neighborhood search operator, boundary constraint handling
	% why it has these modifications, it needs more consideration.
	
	\item CRO application: congnitive radio spectrum allocation\cite{Lam_Power_2013} Skim
	
	\item POA physarum optimization algorithm to solve MEP Minimal exposure problem and general steiner problem\cite{Song_A_2014} %probably needs more read
	\\Connect the food sources in shorter and thicker tubes through less dangerous areas
	
	\item POA to solve steiner tree problem \cite{Liu_Physarum_2015}
	\\Edge-cutting scheme and feedback-adjusting scheme to accelerate the convergence
	
	\item Group size and neocortex ratio were correlated with network efficiency, which is proven from Primates. \cite{Pasquaretta_Social_2014}
		
	\item BBO Biogeography-Based Optimization\cite{Simon_Biogeography_2008}
	\\It is used in practice, but not the fastest, and not general.
	
	\item AMO Animal migration optimization algorithm \cite{Li_Animal_2014}
	\\Move the same direction as the neighbors, remain close to the neighnors, avoid collision

 	\item BMO Bird mating optimizer\cite{Askarzadeh_Bird_2014}
 	\\Male has three groups and female has two groups. Performance looks significant.
 	
	\item  LCA League championship algorithm \cite{Kashan_League_2009}
 	
	\item LCA  for constrained optimization\cite{Kashan_An_2011}
 	\\From unconstrained optimization to constrained optimization

 	\item  FuzzyGES Grouping evolution strategy for fuzzy clustering, unsuspervised \cite{Kashan_An_2013}

	\item l-SL distance based clustering method\cite{Patra_A_2011}
	\\A hybrid scheme with a combination of above two techniques (i.e. leaders and SL method)

 	\item SEP\cite{Gurrutxaga_SEP_2010}	
 
	\item Review of book   Algorithms  to  Live  By:  The  Computer  Science  of  Human  Decisions by  Brian  Christian  and  Tom  Griths,  Henry  Holt, 2016 \cite{Davis_Algorithms_2016}
	\\The  chapters  on  relaxation  techniques  for  solving  optimization  problems,  on  randomized  algorithms,  on  networking  (mostly  TCP  and  buffering  issues),  and  on  game theory  seemed  to  me  excellent,  and  the  chapters  on  explore/exploit,  sorting,  scheduling,  Bayes’  law  and  networking  seemed  to  me fine.  However,  I  think  that  the  chapters  on  optimal stopping,  on  caching,  and  on  overfitting  have  some  serious flaws  that  are  worth  discussing,  and  the  explanation  of  these  is  unavoidably  a  little  lengthy.  Furthermore,  I disagree  substantially  with  their  general  message  about  human  cognition;  and  I  want  to  say  a  few  words  about  that.
	
	\item A  mathematical formalism that captures the general notion of map of a graph and enables its development and manipulation in a semi-automated way\cite{Fionda_Building_2016}
	
	\item Provide a model of a new language .A translator from programming language to first-order logic\cite{Lin_A_2016}	
	\\Don't rely on Hoare’s  loop  invariants

	\item Complexity of the causal ordering problem\cite{Gonalves_A_2016}
	\\ COA to address NP-Hard problem

	\item String loop as holes Qualitative Spatial Reasoning   (QSR)\cite{Cabalar_A_2016}
	\\An optimization to the formal work 

	\item Provide a standard platform for algorithm selection and comparison \cite{Bischl_ASlib_2016}
	\\ For NP-Complete problem. Current SAT competition. Check submitted dataset and suggest a proper algorithm

	\item Generate SAT instances with Community Attachment\cite{Girldez-Cru_Generating_2016}
	\\ High modularity instances

	\item  Merging article affacts H-index\cite{Bevern_H_2016}
	\\ Use Google scholar data, NP- hard problem, a new algorithm, an experiment is developed to see the result

	\item Constraint  Satisfaction  Problem  (CSP)\cite{Veksler_Learning_2016}
	

%\cite{Balasubramanian_Smooth_2016}
%\cite{Balyo_SAT_2016}
%\cite{Bastiaanse_Making_2016}
%\cite{Marcinkiewcz_Serotonin_2016}
 
 




\end{enumerate}
\today


\bibliographystyle{plain}
\bibliography{09212016}
\end{document}