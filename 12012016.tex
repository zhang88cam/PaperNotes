%----------------------------------------------------------------------------------------
%	PACKAGES AND OTHER DOCUMENT CONFIGURATIONS
%----------------------------------------------------------------------------------------

\documentclass[
10pt, % Default font size is 10pt, can alternatively be 11pt or 12pt
a4paper, % Alternatively letterpaper for US letter
onecolumn, % Alternatively onecolumn,twocolumn
portrait % Alternatively portrait,landscape
]{article}

\usepackage{CJKutf8}

%%%%%%%%%%%%%%%%%%%%%%%%%%%%%%%%%%%%%%%%%
% Article Notes
% Structure Specification File
% Version 1.0 (1/10/15)
%
% This file has been downloaded from:
% http://www.LaTeXTemplates.com
%
% Authors:
% Vel (vel@latextemplates.com)
% Christopher Eliot (christopher.eliot@hofstra.edu)
% Anthony Dardis (anthony.dardis@hofstra.edu)
%
% License:
% CC BY-NC-SA 3.0 (http://creativecommons.org/licenses/by-nc-sa/3.0/)
%
%%%%%%%%%%%%%%%%%%%%%%%%%%%%%%%%%%%%%%%%%

%----------------------------------------------------------------------------------------
%	REQUIRED PACKAGES
%----------------------------------------------------------------------------------------

\usepackage[includeheadfoot,columnsep=2cm, left=1in, right=1in, top=.5in, bottom=.5in]{geometry} % Margins

\usepackage[T1]{fontenc} % For international characters
\usepackage{XCharter} % XCharter as the main font

\usepackage{natbib} % Use natbib to manage the reference
\bibliographystyle{apalike} % Citation style

\usepackage[english]{babel} % Use english by default

%----------------------------------------------------------------------------------------
%	CUSTOM COMMANDS
%----------------------------------------------------------------------------------------

\newcommand{\articletitle}[1]{\renewcommand{\articletitle}{#1}} % Define a command for storing the article title
\newcommand{\articlecitation}[1]{\renewcommand{\articlecitation}{#1}} % Define a command for storing the article citation
\newcommand{\doctitle}{\articlecitation\ --- ``\articletitle''} % Define a command to store the article information as it will appear in the title and header

\newcommand{\datenotesstarted}[1]{\renewcommand{\datenotesstarted}{#1}} % Define a command to store the date when notes were first made
\newcommand{\docdate}[1]{\renewcommand{\docdate}{#1}} % Define a command to store the date line in the title

\newcommand{\docauthor}[1]{\renewcommand{\docauthor}{#1}} % Define a command for storing the article notes author

% Define a command for the structure of the document title
\newcommand{\printtitle}{
\begin{center}
\textbf{\Large{\doctitle}}

\docdate

\docauthor
\end{center}
}

%----------------------------------------------------------------------------------------
%	STRUCTURE MODIFICATIONS
%----------------------------------------------------------------------------------------

\setlength{\parskip}{3pt} % Slightly increase spacing between paragraphs

% Uncomment to center section titles
%\usepackage{sectsty}
%\sectionfont{\centering}

% Uncomment for Roman numerals for section numbers
%\renewcommand\thesection{\Roman{section}}
 % Input the file specifying the document layout and structure

%----------------------------------------------------------------------------------------
%	ARTICLE INFORMATION
%----------------------------------------------------------------------------------------

\articletitle{ Multi-objective particle swarm optimization based on global margin ranking} % The title of the article
\articlecitation{\cite{Li_Multi_2017}} % The BibTeX citation key from your bibliography

\datenotesstarted{December 01, 2016} % The date when these notes were first made
\docdate{\datenotesstarted; rev. \today} % The date when the notes were lasted updated (automatically the current date)

\docauthor{QZ} % Your name

%----------------------------------------------------------------------------------------

\begin{document}
\begin{CJK*}{UTF8}{gbsn}
\pagestyle{myheadings} % Use custom headers
\markright{\doctitle} % Place the article information into the header

%----------------------------------------------------------------------------------------
%	PRINT ARTICLE INFORMATION
%----------------------------------------------------------------------------------------

\thispagestyle{plain} % Plain formatting on the first page

\printtitle % Print the title

%----------------------------------------------------------------------------------------
%	ARTICLE NOTES
%----------------------------------------------------------------------------------------

\section{New Concepts} % Unnumbered section

\begin{enumerate}
\item  non-dominated sorting 遗传算法中的一个概念,对多目标问题比较多  
\item  NSGA Non-dominated Sorting Genetic Algorithm
\item  Pareto efficiency
\item Pareto improvement 至少一个目标能够更好,同时保证其他目标不会恶化
\item Pareto frontier For a given system, the Pareto frontier or Pareto set is the set of parameterizations (allocations) that are all Pareto efficient. 
\item 数据的归一化处理
\end{enumerate}

 adaptive fit of a population of candidate solutions to a Pareto front constrained by a set of objective functions

数据归一化的目的是为了把不同来源的数据统一到一个参考系下,这样比较起来才有意义。情况太多了,真的不知道怎么举例子。 

主要看模型是否具有伸缩不变性。

有些模型在各个维度进行不均匀伸缩后,最优解与原来不等价,例如SVM。对于这样的模型,除非本来各维数据的分布范围就比较接近,否则必须进行标准化,以免模型参数被分布范围较大或较小的数据dominate。

有些模型在各个维度进行不均匀伸缩后,最优解与原来等价,例如logistic regression。对于这样的模型,是否标准化理论上不会改变最优解。但是,由于实际求解往往使用迭代算法,如果目标函数的形状太“扁”,迭代算法可能收敛得很慢甚至不收敛。所以对于具有伸缩不变性的模型,最好也进行数据标准化。 


%------------------------------------------------

\section{Questions} % Numbered section

\begin{description}
\item[1]  How Pareto is introduced to multi-objective problems? Is it because the Pareto is dealing with how to improve one object and keep others safe? 
\item[2]  Who firstly brought Pareto into this area?
\item[2]  Is there any other theories that can be used in multi-objective problem?
\item[3]  What is the solution of the MOP?


\end{description}


%------------------------------------------------
\section{Interesting Points}
 Coello bring PSO to multi-objective 2002
  C.A.C.  Coello  ,  M.S.  Lechuga  ,  MOPSO:  a  proposal  for  multiple  objective  particle  swarm  optimization,  in:  Evolutionary  Computation,  2002.  CEC’02.  Pro- 
 ceedings  of  the  2002  Congress  on,  vol.  2,  IEEE,  2002,  pp.  1051–1056  . 

Using Pareto to solve MOP  Pareto-based dominance strategy 
  S.  Biswas,  S.  Das,  P.N.  Suganthan,  C.A.C.  Coello,  Evolutionary  multiobjective  optimization  in  dynamic  environments:  a  set  of  novel  benchmark  functions, 
 in:  2014  IEEE  Congress  on  Evolutionary  Computation  (CEC),  IEEE,  2014,  pp.  3192–3199,  doi:  10.1109/CEC.2014.6900487  . 

%------------------------------------------------

%\section{Discussion/Conclusions Overview}



%------------------------------------------------

\section*{Article Evaluation}
本文主要是针对多目标优化问题中,非支配排序的的效率问题,提出了一种GMR的的算法,能够在目标空间中占主导地位。
考虑了种群的分布,粒子间的通信,调节参数(没有/很少)
GMR可以多多目标的PSO有优化,针对个体的密度问题

多目标问题需要解决的点
simplify  and  accelerate  the  process  of  dominance  relation  assessment.
 gBest  and  pBest  selection  strategy  is  conducted  for  MOPSO.? 为什么? gBest, pBest应该是最大值,为什么还要选择?

MOPSO增加了一个步骤,Trimming external archive?
在选择best值上单多目标问题不一样

大概思路是

选择gBest和pBest时使用不同的方法来避免局部最优解的产生
external archive会使解不归一化

不是很好懂



%----------------------------------------------------------------------------------------
%	BIBLIOGRAPHY
%----------------------------------------------------------------------------------------

\renewcommand{\refname}{Reference} % Change the default bibliography title

\bibliography{11162016} % Input your bibliography file

%----------------------------------------------------------------------------------------
\end{CJK*}
\end{document}